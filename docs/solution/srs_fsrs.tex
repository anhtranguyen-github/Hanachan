\section{Phương pháp học lặp lại ngắt quãng - SRS}

Sau khi xây dựng được cấu trúc nội dung học tập hợp lý, hệ thống Hanachan áp dụng phương pháp học lặp lại ngắt quãng nhằm tối ưu hóa khả năng ghi nhớ dài hạn của người học. Thay vì ôn tập theo lịch cố định, quy trình SRS cho phép hệ thống xác định thời điểm ôn lại phù hợp dựa trên mức độ ghi nhớ thực tế, qua đó giúp người học tập trung vào các nội dung có nguy cơ bị quên cao và giảm thiểu thời gian ôn tập không cần thiết. Đề tài lựa chọn sử dụng thuật toán FSRS (Free Spaced Repetition Scheduler) để triển khai SRS cho các thẻ học trong toàn bộ hệ thống. Khác với các thuật toán truyền thống như SM-2, FSRS mô hình hóa trí nhớ người học thông qua các tham số toán học phức tạp, trong đó mỗi phần tử kiến thức được gắn với một trạng thái ghi nhớ duy nhất và độc lập. Hệ thống quản lý các biến số như độ ổn định của trí nhớ, độ khó bản chất của nội dung và khả năng triệu hồi thông tin tại thời điểm hiện tại để đưa ra lịch trình cá nhân hóa tối ưu.

Trong quá trình tương tác, hệ thống tự động xây dựng hàng đợi học tập dựa trên trạng thái thực tế của dữ liệu, bao gồm các giai đoạn từ làm quen nội dung mới, củng cố kiến thức đang học cho đến bước ôn tập chuyên sâu và tái học các nội dung đã bị lãng quên. Người học thực hiện đánh giá mức độ ghi nhớ sau mỗi lần tương tác thông qua các mức phản hồi định lượng, từ đó công cụ tính toán sẽ ngay lập tức cập nhật lại các tham số lưu giữ và xác định thời điểm xuất hiện lần tiếp theo. Cách tiếp cận này cho phép tiến trình học được điều chỉnh trực tiếp theo khả năng tiếp thu của từng cá nhân mà không phụ thuộc vào một lịch trình chung cứng nhắc. Một đặc điểm quan trọng trong triển khai của dự án là trạng thái FSRS được gán trực tiếp cho từng đơn vị tri thức cụ thể như Hán tự, từ vựng và cấu trúc ngữ pháp thay vì chỉ gán cho một thực thể hiển thị đơn thuần. Việc tương tác với một nội dung tại bất kỳ phân hệ nào cũng sẽ dẫn đến việc cập nhật trạng thái ghi nhớ trên toàn hệ thống, giúp duy trì tiến trình học nhất quán và tránh việc ôn tập trùng lặp không cần thiết. Sự tích hợp sâu của FSRS không chỉ giúp chuyển đổi hình thức học từ thụ động sang chủ động mà còn giúp người học duy trì được nhịp độ ổn định và bền vững hơn theo thời gian.

\section{Giải pháp học tích hợp với AI}

Trong hệ thống Hanachan, trí tuệ nhân tạo được tích hợp sâu rộng nhằm hỗ trợ học tập theo hướng tương tác và cá nhân hóa, đóng vai trò như một trợ lý thông minh có khả năng đồng hành cùng người dùng xuyên suốt lộ trình. Giải pháp AI được thiết kế xoay quanh mô hình hội thoại chuyên biệt cho ngôn ngữ Nhật, sở hữu năng lực phân tích ngôn ngữ tự nhiên, liên kết tri thức hệ thống và điều chỉnh phản hồi phù hợp với ngữ cảnh sử dụng thực tế. Khi người dùng nhập nội dung tương tác, hệ thống trước hết thực hiện quy trình nhận diện mục đích để phân loại chính xác các truy vấn về cấu trúc câu, giải thích ngữ pháp hoặc yêu cầu luyện tập giao tiếp. Đối với các yêu cầu phân tích chuyên sâu, trợ lý AI sẽ kích hoạt các bộ xử lý ngôn ngữ lõi để thực hiện tách từ, nhận diện từ loại và xác định các mẫu kiến trúc xuất hiện trong văn bản. Kết quả của quy trình này được sử dụng làm cơ sở để đưa ra chiến lược phản hồi tối ưu, đảm bảo nội dung trả lời luôn đi thẳng vào trọng tâm và phù hợp với trình độ của người học.

Dựa trên dữ liệu phân tích ngữ cảnh, hệ thống tiến hành liên kết tự động câu đầu vào với các đơn vị kiến thức đang tồn tại trong cơ sở dữ liệu như từ vựng hoặc ngữ pháp cụ thể. Quá trình này giúp trợ lý AI hiểu rõ bối cảnh học tập hiện tại của người dùng, từ đó tránh việc giải thích lặp lại các nội dung đã được ghi nhận là đạt mức độ thông thạo cao. Phản hồi của hệ thống được xây dựng theo nguyên tắc cung cấp lượng thông tin vừa đủ và đúng thời điểm để hỗ trợ tối đa cho việc ghi nhớ mà không gây gián đoạn luồng suy nghĩ chính. Bên cạnh đó, các hành động học tập mở rộng được lồng ghép tinh tế thông qua các điểm tương tác bổ trợ, cho phép người dùng chủ động đào sâu các thông tin liên quan khi có nhu cầu thực tế. Hệ thống cũng liên tục ghi nhận các tương tác học tập ở mức độ tiếp xúc để điều chỉnh hành vi của trợ lý trong các lần làm việc tiếp theo, chẳng hạn như việc tăng dần độ khó của từ vựng hoặc tinh chỉnh phong cách hướng dẫn. Nhờ việc tích hợp AI theo hướng cá nhân hóa và gắn liền với ngữ cảnh thực tế, nền tảng không chỉ hỗ trợ giải đáp thắc mắc tức thời mà còn tạo ra một môi trường học tập chủ động, giúp người học vận dụng kiến thức một cách tự nhiên và hiệu quả.

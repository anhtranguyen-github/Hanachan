\section{Giải pháp học tích hợp với AI}

Trong hệ thống Hanachan, trí tuệ nhân tạo được tích hợp nhằm hỗ trợ học tập theo hướng tương tác và bóc tách nội dung, đóng vai trò như một trợ lý tri thức thông minh. Giải pháp AI được thiết kế xoay quanh mô hình hội thoại chuyên biệt cho ngôn ngữ Nhật, sở hữu năng lực phân tích ngôn ngữ tự nhiên, liên kết tri thức hệ thống và điều chỉnh phản hồi phù hợp với ngữ cảnh nội dung đang trao đổi. Khi người dùng nhập nội dung tương tác, hệ thống thực hiện quy trình suy luận bằng LLM để phân loại các truy vấn về cấu trúc câu, giải thích ngữ pháp hoặc bóc tách từ vựng. Đối với các yêu cầu phân tích chuyên sâu, trợ lý AI sẽ kích hoạt các công cụ xử lý ngôn ngữ để thực hiện nhận diện thực thể (Entity Detection) và xác định các mẫu kiến thức xuất hiện trong văn bản. Kết quả của quy trình này được sử dụng làm cơ sở để đưa ra phản hồi chính xác, đảm bảo nội dung trả lời luôn đi thẳng vào trọng tâm kiến thức.

Dựa trên dữ liệu phân tích ngữ cảnh, hệ thống tiến hành liên kết tự động nội dung trao đổi với các đơn vị kiến thức (Knowledge Units) đang tồn tại trong cơ sở dữ liệu. Quá trình này giúp trợ lý AI cung cấp các giải thích có chiều sâu và chính xác về mặt học thuật bằng cách truy vấn trực tiếp từ kho tri thức gốc. Phản hồi của hệ thống được xây dựng theo nguyên tắc cung cấp lượng thông tin hữu ích và đúng trọng tâm để hỗ trợ tối đa cho việc hiểu sâu kiến thức. Bên cạnh đó, các hành động học tập mở rộng được lồng ghép thông qua các điểm tương tác bổ trợ (CTA), cho phép người dùng chủ động xem chi tiết các thông tin liên quan như bộ thủ của Kanji hay ví dụ của ngữ pháp. Bằng cách tập trung vào việc bóc tách và liên kết tri thức khách quan, nền tảng tạo ra một môi trường học tập chuyên sâu, giúp người học làm chủ kiến thức tiếng Nhật một cách có hệ thống và hiệu quả.
